%%%%%%%%%%%%%%%%%%%%%%%%%%%%%%%%%%%%%%%%%%%%%%%%%%%%%%%%%%%%%%%%%%%%%%%%%%%%%%%%%%%%%%%%%%
\section{Conclusions and Future Work} \label{sect:conclusions}                           %
%%%%%%%%%%%%%%%%%%%%%%%%%%%%%%%%%%%%%%%%%%%%%%%%%%%%%%%%%%%%%%%%%%%%%%%%%%%%%%%%%%%%%%%%%%

The core goal for ApiNATOMY is to put clinicians, pharmacologists, basic
scientists and other biomedical experts in direct control of physiology
knowledge management (e.g. in support of integrative goals outlined in~\cite{hunter_vision_2010}).
As the domain of physiology deals with processes across multiple anatomical scales,
the schematic ApiNATOMY approach provides a more flexible and customizable
depiction of process participants, and the routes they undertake, compared to
conventional methods of anatomy navigation that constrain visualization to
regional views of very detailed and realistically proportioned 3D models (such as
Google Body~\cite{ZygoteBody}).
In this paper, we presented our initial results in the development of a generic
tool that creates an interactive topological map of physiology communication
routes. These routes are depicted in terms of (i) treemaps derived from standard
reference anatomy ontologies, as well as (ii) networks of cardiovascular and
neural connections that link tiles within these treemaps. These topological maps,
also known as circuitboard schematics, set the stage for the visual management
of complex genomic and drug-related data in terms of the location of gene
products and the route taken by molecules that interact with them. While the
implementation of our tool is still in its early stages, we have already started
taking steps in preparation for future developments, supporting:

\begin{itemize}
  \item the visually-enhanced construction of mathematical models in systems 
biology (e.g., as discussed in~\cite{de_bono_integrating_2012}),
  \item the collaborative graphical authoring of routes of physiology
communication (e.g., brain circuits) and, crucially,
  \item the automated discovery of transport routes given (i) a fixed-
location receptor and (ii) its corresponding ligand, found elsewhere
in the body.
\end{itemize}

Above all, our aim is to ensure that ApiNATOMY is easy to use for biomedical
professionals, and available across a wide range of platforms, to foster
collaborative exchange of knowledge both within, and between, physiology
communities.

