%%%%%%%%%%%%%%%%%%%%%%%%%%%%%%%%%%%%%%%%%%%%%%%%%%%%%%%%%%%%%%%%%%%%%%%%%%%%%%%%%%%%%%%%%%
\section{Introduction}                                                                   %
%%%%%%%%%%%%%%%%%%%%%%%%%%%%%%%%%%%%%%%%%%%%%%%%%%%%%%%%%%%%%%%%%%%%%%%%%%%%%%%%%%%%%%%%%%

An ontology consists of a set of terms, and their relations, representing a specific domain of knowledge. They are created and maintained by knowledge domain experts, and are used as computer-readable taxonomies by software tools to support knowledge management activities in that domain.

The complexity of some of the ontologies in current use, as well as the complexity of handling semantic metadata that annotate third party resources with ontology terms (e.g. as described in~\cite{BHW+11}), has generated considerable demand for effective visualization in the design, authoring, navigation and management of (i) ontology-based knowledge and (ii) semantic metadata that make use of ontologies.

In response to the above demand, a number of generic ontology visualization tools have been developed to assist knowledge acquisition, browsing and maintenance of ontologies~\cite{KHL+07}. Such tools, however, put considerable and unrealistic demands on the users' familiarity and expertise in both (i) semantic web technologies and (ii) the design principles of ontologies. It is unlikely that a user with expertise in the \emph{domain} of an ontology also has expertise in the \emph{technologies} that handle ontologies.

The domain of biomedical physiology is a case in point. Physiology experts deal with complex biophysical operations, across multiple spatial and temporal scales, which they represent in terms of the transfer of energy from one form to another and/or from one anatomical location to another. Different kinds of descriptions of these biophysical operations are produced by different disciplines in biomedicine. For instance, (i) a medical doctor may describe the mechanism by which a stone in the ureter causes damage in the kidney; (ii) a pharmacologist may depict the process by which a drug absorbed from gut transits to the hip joints where it reduces inflammation; (iii) a molecular geneticist may trace the anatomical distribution of the expression of particular gene to understand the cause of a skeletal malformation; and, (iv) a bioengineer may build a mathematical model to quantify the effect of hormone production by the small intestine on the production of bile by the liver. These descriptions take diverse forms, ranging from images and free text (e.g., a paper in a journal) to XML documents bearing well-defined data (e.g. from a clinical trial) or sets of model variables and related equations (which could be used for as input for a simulation tool).

Automating the discovery of relationships, in terms of physiological meaning, between the above types of description is a key goal to the physiology community. To that end, this community is investing considerable effort in building ontologies for the annotation and semantic management of resources describing physiology. A number of reference ontologies have been created to represent the various entities required to describe physiology, including gene products~\cite{Bla+13}, chemical entities~\cite{HMD+13}, cells~\cite{BRA05} and gross anatomy~\cite{RM03}. Cumulatively, these reference ontologies consist of hundreds of thousands of terms, such that the volume of semantic metadata arising from annotation of resources with these terms is considerable. However, conventional technology for the management of ontologies and metadata is not usefully accessible to physiology experts.

The ApiNATOMY effort has emerged to provide intuitive graphical interface for managing ontologies and semantic metadata relevant to physiology. In this paper, we present a web-based ApiNATOMY environment for physiology experts to navigate through circuitboard visualizations of body components and their physiological connections across scales. In particular, we present a tool prototype that visualizes schematics of ontology-based knowledge about body parts and their cardiovascular and neural connections. Graphical renderings of semantic annotations to (i) gene product data and (ii) process models are overlaid onto these schematics, in support of biomedical knowledge management use cases discussed in the next Section.

This paper is structured as follows: first, we give an overview of the ontology, metadata and data resources that we focused on for this prototype, and outline usage scenarios that motivate our work (\cref{sect:motivation}).  We then discuss representation and visualization methods applied to arrange and display relevant resources (\cref{sect:visualization,sect:visualization2}). In \cref{sect:implementation}, we describe the graphical tool prototype that implements these methods. Furthermore, in \cref{sect:relatedWork}, we overview related methodologies, efforts and techniques in the field. Finally, we conclude the paper with a discussion of the anticipated implications of this tool, as well as and planned future work.
