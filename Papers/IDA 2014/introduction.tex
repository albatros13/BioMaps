%%%%%%%%%%%%%%%%%%%%%%%%%%%%%%%%%%%%%%%%%%%%%%%%%%%%%%%%%%%%%%%%%%%%%%%%%%%%%%%%%%%%%%%%%%
\section{Introduction}                                                                   %
%%%%%%%%%%%%%%%%%%%%%%%%%%%%%%%%%%%%%%%%%%%%%%%%%%%%%%%%%%%%%%%%%%%%%%%%%%%%%%%%%%%%%%%%%%

Knowledge of physiology is extensive and complex. To provide software support for using and manipulating physiology data, formalization of the knowledge is required. An \emph{ontology} consists of a set of terms, and their relations, representing a specific domain of knowledge. They are created and maintained by knowledge domain experts, and are used as computer-readable taxonomies by software tools to support knowledge management activities in that domain.
When knowledge is formalised in this way, it is possible to record explicit descriptions of data elements in the relevant domain using ontologies; this is the process of semantic annotation or the generation of \emph{semantic metadata}.

The complexity of some of the ontologies in current use, as well as the complexity of handling semantic metadata that annotate third party resources with ontology terms (e.g. as described in~\cite{BHW+11}), has generated considerable demand for effective visualization in the design, authoring, navigation and management of (i) ontology-based knowledge and (ii) semantic metadata based on ontologies.

In response to the above demand, a number of generic ontology visualization tools have been developed to assist knowledge acquisition, browsing and maintenance of ontologies~\cite{KHL+07}. Such tools, however, put considerable and unrealistic demands on the users' familiarity and expertise in both semantic web technologies and the design principles of ontologies. It is unlikely that a user with \emph{domain} expertise also has expertise in the \emph{technologies} used to manage ontologies.

For example, physiology experts deal with complex biophysical operations across multiple spatial and temporal scales, which they represent in terms of the transfer of energy from one form to another and/or from one anatomical location to another. Different kinds of descriptions of these biophysical operations are produced by different disciplines in biomedicine. For instance, (i) a medical doctor may describe the mechanism by which a stone in the ureter causes damage in the kidney; (ii) a pharmacologist may depict the process by which a drug absorbed from gut transits to the hip joints where it reduces inflammation; (iii) a molecular geneticist may trace the anatomical distribution of the expression of a particular gene to understand the cause of a skeletal malformation; and, (iv) a bio-engineer may build a mathematical model to quantify the effect of hormone production by the small intestine on the production of bile by the liver. These descriptions take diverse forms, ranging from images and free text (e.g., a journal paper) to models bearing well-defined data (e.g. from a clinical trial) or sets of mathematical equations (which might be used as input for a simulation tool).

Automating the discovery of relationships, in terms of physiological meaning, between these kinds of description is a key goal to the physiology community. Therefore, this community is investing considerable effort in building ontologies for the annotation and semantic management of such resources. For example, a number of reference ontologies have been created to represent gene products~\cite{Bla+13}, chemical entities~\cite{HMD+13}, cells~\cite{BRA05} and gross anatomy~\cite{RM03}. Together, these ontologies consist of hundreds of thousands of terms, such that the volume of semantic metadata arising from resource annotation is considerable. Unfortunately, conventional technology for the management of ontologies and metadata is not usefully accessible to physiology experts, as they involve unfamiliar, abstract technicalities.
%%
Having to become technically proficient with such technology is a burden few physiology experts can bear without losing touch with their long term goals.
%%%% (MH: Rewritten and shortened above)
%The pitfall for the physiology expert wishing to take advantage of semantic technology
%is that they have to become technically proficient and lose sight of their long term goal
%and lose touch with their own knowledge, having to struggle with unfamiliar, abstract technicalities.
%%
Rather, domain experts should be able to manage data based on a familiar perspective, in which its meaning is made explicit in terms of the expert's own knowledge; a long standing challenge in knowledge engineering.

The present work builds on past efforts on RICORDO~\cite{BHW+11}. Those efforts have led to the design of backend software for the management of physiology metadata, including physiology models, genomics data, and radiology and histology images. This software comprises four notionally separable components: (a) expert knowledge, (b) a data repository, (c) a metadata repository and (d) an ontology knowledge base able to perform reasoning over ontology relationships.
%%%% (MH: not needed)
%These components need to be tied together so that an ontology which corresponds to the expert
%knowledge can be used in the annotation of data and that browsing data repositories is done by
%proxy through the searching of a metadata store using ontologies and therefore reflecting expert knowledge.
%%
The ApiNATOMY effort has emerged to provide an intuitive graphical frontend to RICORDO. In this paper, we present a web-based ApiNATOMY environment, allowing physiology experts to navigate through circuitboard visualizations of body components, and their cardiovascular and neural connections, across different scales. Overlaid on these schematics are graphical renderings of organs, neurons and gene products,
as well as mathematical process models, in support of biomedical knowledge management use cases discussed in the next section.

The remainder of the paper is structured as follows: \cref{sect:motivation} gives an overview of the \mbox{ontology-,} metadata- and data-resources that we focused on for the ApiNATOMY prototype, and outline key use-case scenarios that motivate our work. \Cref{sect:visualization,sect:visualization2} then discuss the visualization techniques we applied to arrange and display those resources. \Cref{sect:implementation} provides some insight into the implementation of the prototype. Finally, \cref{sect:relatedWork,sect:conclusions} offer an overview of related efforts in the field, and conclude the paper with a discussion of the anticipated implications of our tool, as well as planned future work.
