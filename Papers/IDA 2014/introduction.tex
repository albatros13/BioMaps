\section{Introduction}

An ontology represents a set of concepts and their interrelations in a specific domain. 
Ontologies are created and maintained by domain experts and used as reference taxonomies by other professionals as well as software tools supporting their activities. The need for effective ontology visualization for design, management, and browsing has arisen.
Consequently, many ontology visualization tools have been developed to assist semantic knowledge acquisition and maintenance~\cite{KHL+07}.
However, the major part of this knowledge remains unaccessible for people without expertise in semantic web technologies and understanding of design principles for specific types of ontologies. We aim at addressing this issue in biomedical domain by generating body plans and schematics for ongoing biomedical processes from existing ontological resources.

In this paper, we present an integrated web-based environment for biomedical experts to navigate through schematically represented body tissues, browse, analyze and extend associated data resources and reason about multi-scale processes in selected regions of human body at any level of detail: from body parts and organs, to cells, molecules, proteins, and gene expressions. More specifically, we present a tool prototype that visualizes body parts, cardiovascular and neural connections, mathematical models of ongoing processes, static and dynamic 3D models, protein structures,... 

First, we overview biomedical data resources we aim at providing access to and outline several usage scenarios that motivate our work (Section~\ref{sect:motivation}).
Then we discuss data representation and visualization methods used to arrange and display relevant resources (Section~\ref{sect:visualization}). We describe a graphical tool prototype (Section~\ref{sect:implementation}).
Furthermore, we overview related efforts and techniques (Section~\ref{sect:relatedWork}). 
Finally, we conclude the paper and discuss future work (Section~\ref{sect:conclusions}).
