%%%%%%%%%%%%%%%%%%%%%%%%%%%%%%%%%%%%%%%%%%%%%%%%%%%%%%%%%%%%%%%%%%%%%%%%%%%%%%%%%%%%%%%%%%
\section{Use Cases and Data Resources} \label{sect:motivation}                           %
%%%%%%%%%%%%%%%%%%%%%%%%%%%%%%%%%%%%%%%%%%%%%%%%%%%%%%%%%%%%%%%%%%%%%%%%%%%%%%%%%%%%%%%%%%

In this section, we briefly discuss a core use case for the ApiNATOMY application: the generation of
interactive schematics in support of genomics and drug discovery studies. We introduce some of the
key ontology- and data-resources required in this case. In so doing, we set the stage for an
exposition of our early-stage results in the ApiNATOMY application effort.

The domains of genomics and drug discovery are heavily dependent on physiology knowledge, as both
domains take into account the manufacture of proteins in different parts of the body and the
transport of molecules that interact with those proteins, such as drugs, nutrients, and other
proteins.
%%
We aim to provide an interactive, schematic overview of data resources important to these domains.
This includes gene expression data (e.g.,~\cite{EBI}), and data on the transport routes taken by
molecular interactors (e.g.,~\cite{HMC+13}). Such data may be usefully depicted in the form of a
physiology \emph{circuitboard}.

In ApiNATOMY, a physiology circuitboard schematic consists of an \emph{anatomical treemap} and an
overlay of \emph{process graphs}. Our earlier prototypes~\cite{BGS12,KBK14} presented treemaps of
the Foundational Model of Anatomy (FMA) ontology~\cite{RM03}. Nesting of one treemap tile inside
another indicated that the term associated with the child tile is either a mereotopological
\emph{part} or a \emph{subclass} of the term associated with the parent tile. Our newest prototype
also adopts this convention.

\begin{figure}%
	\centering%
	\begin{minipage}[b]{.52\linewidth}
		\centering%
		\subfigure[Initial view of ApiNATOMY]{
			\includegraphics[width=\linewidth]{images/screenshot-main.png}
			\label{fig:24tiles}
		}
	\end{minipage}\hskip4mm
	\begin{minipage}[b]{.44\linewidth}
		\centering%
		\subfigure[Longitudinal section through the male human body, justifying the layout]{
			\includegraphics[width=\linewidth]{images/tilemap-cylinder.png}
			\label{fig:tilemap-cylinder}
		}
	\end{minipage}\vskip2mm
	\caption{The main 24-tile layout of the ApiNATOMY circuitboard}
	\label{fig:treemaps}
\end{figure}

The ApiNATOMY graphical user interface (\cref{fig:24tiles}) supports user interaction with
circuitboard schematics via point-and-click navigation of the treemap content. The upper level of
the anatomical treemap is arranged to resemble the longitudinal section through the middle of the
human body (\cref{fig:tilemap-cylinder}). Each of the organs in the plan is composed of multiple
tissues and sub-organs. The GUI supports data filtering across multiple levels and contextual
zooming into selected areas.

This type of interaction extends also to the overlayed process graphs. These graphs project routes
of blood flow processes linking different regions of the human body ---using data generated
in~\cite{deB11}---, as well as transport processes along neurons of the central nervous system
(i.e., the brain and spinal cord) --- using data obtained via the Neuroscience Information
Framework~\cite{Gar+08}.

The ApiNATOMY GUI is built from inception as a three-dimensional environment. This facilitates
interaction not only with 3D renderings of the circuit boards themselves, but also with a wide range
of geometry/mesh formats for volumetric models of biological structure across scales. For instance,
it is already possible to overlay Wavefront \texttt{.obj} data from BodyParts3D~\cite{MFT+09} as
well as \texttt{.swc} data provided by \texttt{neuromorpho.org}~\cite{Asc06}. Easy access to such
visual resources is critical to the understanding of long-range molecular processes in genomics and
drug discovery research.

In the next two sections, we discuss our techniques for
constraining treemap layouts to generate stable anatomical treemaps (\cref{sec:treemaps}),
designing and overlaying physiological communication routes for the cardiovascular and
	neural systems (\cref{sec:process-graphs,sec:process-graphs-example}), and
depicting three-dimensional models of organs and protein architecture diagrams for the anatomical
	overview of gene expression data (\cref{sect:visualization2}).




